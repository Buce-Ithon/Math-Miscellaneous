\documentclass{Math_Note}

\title{Speedrun Quadratic Residue \\ \small{Cryptography Breaks}}
\author{Buce-Ithon}
\newdateformat{mydate}{\twodigit{\THEDAY}{ }\shortmonthname[\THEMONTH], \THEYEAR}
\date{\today}

\begin{document}

%Title page
\maketitle

%Content
\newpage
\tableofcontents
\newpage

%Beginning
\section{Unique Section}

The definition of \underline{quadratic remainder} comes from the expansion of perfect square numbers($x^2$) in the integer system into the multiplicative group modulo $q$.

\begin{df}
    \textbf{Quadratic Residue} \par
    $\forall\ prime\ number\ p$,$a \in \mathbb{Z}_{p}$, $a$ is a quadratic residue modulo $p$ if $\exists\ integer\ x$ such that $x^2 \equiv a \pmod{p}$.
    Otherwise, $a$ is a quadratic non-residue modulo $p$.
\end{df}

\begin{df}
    \textbf{Legendre Symbol} \par
    $\forall\ odd\ prime\ number\ p$,$a \in \mathbb{Z}_{p}$, the Legendre symbol is defined as:
    \[
        \left(\frac{a}{p}\right) = 
        a^{\frac{p-1}{2}}\ mod\ p\equiv
        \begin{cases}
            1 & \text{if $a$ is a quadratic residue modulo $p$} \\
            -1 & \text{if $a$ is a quadratic non-residue modulo $p$} \\
            0 & \text{if $a \equiv 0 \pmod{p}$}
        \end{cases}
    \]
\end{df}

\begin{prp}
    \begin{enumerate}
        \item $\left(\frac{ab}{p}\right)=\left(\frac{a}{p}\right)\left(\frac{b}{p}\right)$
        \item If $a \equiv b \pmod{p}$, then $\left(\frac{a}{p}\right)=\left(\frac{b}{p}\right)$
    \end{enumerate}
\end{prp}

\begin{thm}
    \textbf{Gauss Law of Quadratic Reciprocity} \par
    $\forall\ odd\ prime\ numbers\ p$ and $q$, the Legendre symbol satisfies:
    \[
        \left(\frac{p}{q}\right)\left(\frac{q}{p}\right)=(-1)^{\frac{p-1}{2}\frac{q-1}{2}}
    \]
\end{thm}

\begin{df}
    \textbf{Jacobi Symbol} \par
    $\forall\ odd\ integer\ n$ and $a \in \mathbb{Z}_{n}$, $gcd(n,a)=1$, the Jacobi symbol is defined as:
    \[
        \left(\frac{a}{n}\right) = 
        \prod_{i=1}^{k}\left(\frac{a}{p_i}\right)^{e_i}
    \]
    where $n=\prod_{i=1}^{k}p_i^{e_i}$ is the prime factorization of $n$.
\end{df}

Moreover, 2 important theorems should be mentioned here.

\begin{thm}
    \text{Euler's Theorem} \par
    $\forall\ integer\ a$ and $n$, $gcd(a,n)=1$, then:
    \[
        a^{\phi(n)}\equiv 1\pmod{n}
    \]
\end{thm}

\begin{thm}
    (Collary)\text{Fermat's Little Theorem} \par
    $\forall\ prime\ number\ p$ and integer $a$, then:
    \[
        a^{p-1}\equiv 1\pmod{p}
    \]
\end{thm}

Last but not least, let's come back to cryptography, looking at some applications of Legendre symbol.

\begin{thm}
    \textbf{Solovay-Strassen Primality Test} \par
    \begin{enumerate}
        \item $\forall\ odd\ prime\ p$, then $\forall\ integer\ a$, we have $\left(\frac{a}{p}\right) \equiv a^{\frac{p-1}{2}}\ mod\ p$.
        \item $\forall\ odd\ composite\ n$, then there are at least 50\% integer $a$, s.t. $\left(\frac{a}{n}\right) \equiv a^{\frac{n-1}{2}}\ mod\ n$ is false.
    \end{enumerate}
\end{thm}

\end{document}