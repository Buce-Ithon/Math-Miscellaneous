\documentclass{Math_Note}

\title{A Tiny Tip in Fraction2Polynomial with Taylor Series Base}
\author{Buce-Ithon}
\newdateformat{mydate}{\twodigit{\THEDAY}{ }\shortmonthname[\THEMONTH], \THEYEAR}
\date{\today}

\begin{document}

%Title page
\maketitle

%Content
\newpage
\tableofcontents
\newpage

%Beginning
\section{Inspiration - Learning from a defination}
\begin{df}
    Define the \textbf{Bernoulli number} as
    \begin{equation}
        \frac{x}{e^{x}-1} = \Sigma_{n=0}^{\infty}\frac{B_{n}}{n!}x^{n}
    \end{equation} 
\end{df}    
Expanding $e^{x}$ with Taylor series $e^{x}=1+x+\frac{x^{2}}{2!}+\frac{x^{3}}{3!}+\cdots$, 

then we have
\begin{equation}
    \begin{split}
        \frac{x}{e^{x}-1} &= \frac{x}{x+\frac{x^{2}}{2!}+\frac{x^{3}}{3!}+\cdots} = \frac{1}{1+\frac{x}{2!}+\frac{x^2}{3!}+\cdots} \\
        &=1 - \left(\frac{x}{2!}+\frac{x^2}{3!}+\cdots\right) + \left(\frac{x}{2!}+\frac{x^2}{3!}+\cdots\right)^2-\cdots
    \end{split}
\end{equation}

and then
\begin{equation}
    \begin{split}
        &B_{0} = 1, B_{1} = -\frac{1}{2}, B_{2} = \frac{1}{6}, B_{4} = -\frac{1}{30}, B_{6} = \frac{1}{42}, \\
        &B_{8} = -\frac{1}{30}, B_{10} = \frac{5}{66}, B_{12} = -\frac{691}{2730}, B_{14} = \frac{7}{6}
    \end{split}
\end{equation}

%Tool-kit
\section{Tip or Trick - Tool-kit from it}
Actually, the step (2) can be abstracted as a mathematical tip as follow:
\begin{equation}
    \frac{1}{1+A} = 1 - A + A^{2} -A^{3} + \cdots = \Sigma_{n=0}^{\infty}A^{n}\cdot\left(-1\right)^{n}
\end{equation}
\begin{pf}
    \begin{equation}
        \begin{split}
            {\left(1+A\right)}\cdot&{\left(1 - A + A^{2} -A^{3} + \cdots\right)} \\
            &= {1 - A + A^{2} -A^{3} + \cdots} + {A - A^{2} + A^{3} - A^{4} + \cdots}\\
            &= 1           
        \end{split}
    \end{equation}
\end{pf}

Application: Simplify fractions $\frac{1}{1+A}$ to polynomials.

%Taylor series base
\section{Essence - Taylor Series}
The essencial of this transformation is Taylor Series of function $f(x) = \frac{1}{1+x}$ in $x=0$. \\

\begin{equation}
    \begin{split}
        f(x) &= \Sigma_{n=0}^{\infty} \frac{f^{(n)}(x_{0})}{n!}\left(x-x_{0}\right)^{n} \\
        In \quad &x_{0} = 0, \\
        f(x) &= \frac{1}{1+x} \\
        &= 1-x+x^{2}-x^{3}+\cdots = \Sigma_{n=0}^{\infty}(-1)^{n}x^{n}
    \end{split}
\end{equation}

%Exercise
\section{Think twice - exercise questions}
Taylor Series is an approximation in $x = x_{0} \left(= 0 \quad here\right)$, but 
why it "seemly" fits all $x$ values in $\mathbb{R}$ in this tip? 
Is there any other examples like that, or just to say the Taylor series is always accurate?
\end{document}