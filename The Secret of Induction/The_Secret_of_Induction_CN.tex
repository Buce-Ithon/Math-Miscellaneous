\documentclass{Math_Note}
\usepackage{ctex}

\title{The Secret of Induction \\ \small ------A Courese Note of MIT6.042 E2 \\ 归纳法之谜 \small ------A Courese Note of MIT6.042 E2}
\author{Buce-Ithon}
\newdateformat{mydate}{\twodigit{\THEDAY}{ }\shortmonthname[\THEMONTH], \THEYEAR}
\date{\today}

\begin{document}

%Title page
\maketitle

%Content
\newpage
\tableofcontents
\newpage

%Beginning
\section{Introduction 介绍}
Mathmatical induction is a common method, or exactly an important idea, in numerous mathematical proofs. However, how much do we really know about it?
How deeply do we understand it? 

数学归纳法是出现在众多数学证明中的一种常用方法,或者更确切地说是一种重要思想。 然而,我们对这种方法真正了解多少呢?我们对它又有着多深刻的理解?

In this article, I will try to explain the secret of induction, and show you how to use it in a more flexible way with 
deeper comprehension.

在这篇文章中,我将尝试举例解释归纳的原理,并带领读者一步步揭开这种思想的内蕴之谜,从而使我们能以更灵活的方式、更深入的理解来使用它。

\section{Two Simple Examples 两个简单的例子}
Before we starting, let's describe the basic concept of induction. Induction is a method to prove a statement for all positive integers or to say situations: 

在正式开始之前,让我们先来回顾一下归纳法的基本概念————归纳法是一种递归式地证明所有(正)整数种陈述或命题的方法:

\begin{mdframed}
\begin{thm}[Induction Axiom 归纳公理]
    Let $P(n)$ be a statement or predicate involving a positive integer $n$. If 令$P(n)$表示一个涉及正整数$n$的陈述或命题。若
    \begin{enumerate}
        \item $P(1)$ is true, and 基础情况$P(1)$是正确的,且
        \item For all positive integers $k$, if $P(k)$ is true, then $P(k+1)$ is true, 对于所有正整数$k$,如果$P(k)$是正确的,那么$P(k+1)$也是正确的,
    \end{enumerate}
    then $P(n)$ is true for all positive integers $n$. 则$P(n)$对于所有正整数$n$都是正确的。
\end{thm}
\end{mdframed}

Now let's see two simple examples to show the basic idea of induction.

现在,让我们看两个简单的例子以展示归纳法的基本思想。

\begin{prb}
    Prove that $\forall n \in \mathbb{N}$, the sum of the first $n$ positive integers $\Sigma_{k=1}^{n} k=\frac{n(n+1)}{2}$.
    证明对于所有$n \in \mathbb{N}$,前$n$个正整数的和$\Sigma_{k=1}^{n} k=\frac{n(n+1)}{2}$。
\end{prb}

\begin{sol}
    Let $P(n)$ be the statement/predicate that the sum of the first $n$ positive integers: $\Sigma_{k=1}^{n} k=\frac{n(n+1)}{2}$. We will prove $P(n)$ is true for all positive integers $n$.

    令$P(n)$表示这样一个陈述:一个$n$个正整数的和$\Sigma_{k=1}^{n} k=\frac{n(n+1)}{2}$。我们将证明对于所有正整数$n$,$P(n)$是正确的。

    \textbf{Base Case:} $P(1)$ is true, since the sum of the first positive integer is $1 = \frac{1(1+1)}{2}$.
    
    \textbf{基础情况:} $P(1)$是正确的,因为第一个正整数的和是$1 = \frac{1(1+1)}{2}$。

    \textbf{Inductive Step:} Assume that $P(k)$ is true for some positive integer $k$, that is, the sum of the first $k$ positive integers is $\frac{k(k+1)}{2}$. We need to show that $P(k+1)$ is true, that is, the sum of the first $k+1$ positive integers is $\frac{(k+1)(k+2)}{2}$.

    \textbf{归纳步骤:} 假设对于某个正整数$k$,$P(k)$是正确的,即$k$个正整数的和是$\frac{k(k+1)}{2}$。我们需要证明$P(k+1)$是正确的,即$k+1$个正整数的和是$\frac{(k+1)(k+2)}{2}$。

    By the inductive hypothesis, the sum of the first $k+1$ positive integers is 

    由归纳假设,$k+1$个正整数的和是
    \begin{align*}
        1+2+\cdots+k+(k+1) &= \frac{k(k+1)}{2} + (k+1)\\
        &= \frac{k(k+1)+2(k+1)}{2}\\
        &= \frac{(k+1)(k+2)}{2}.
    \end{align*}
    Thus, $P(k+1)$ is true.

    因此,$P(k+1)$是正确的。

    By the inductive axiom, $P(n)$ is true $\forall n \in \mathbb{N}$.

    由归纳公理,$P(n)$对于所有$n \in \mathbb{N}$都是正确的。
\end{sol}

The next problem is a little bit more complicated, but we still can handle it with this axiom.

下一个问题要稍微复杂一些,不过我们同样可以用归纳公理来处理它。

\begin{prb}
    Prove that $\forall n \in \mathbb{N}$, $3 \mid n^{3}-n$.
    证明$\forall n \in \mathbb{N}$,$3 \mid n^{3}-n$。
\end{prb}

\begin{sol}
    Let $P(n)$ be the statement/predicate that $3 \mid n^{3}-n$. We will prove $P(n)$ is true for all positive integers $n$.

    令$P(n)$表示这样一个陈述:$3 \mid n^{3}-n$。我们将证明对于所有正整数$n$,$P(n)$是正确的。

    \textbf{Base Case:} $P(1)$ is true, obviously $1^{3}-1=0$ is divisible by 3.

    \textbf{基础情况:} $P(1)$是正确的,显然$1^{3}-1=0$是3的倍数。

    \textbf{Inductive Step:} Assume that $P(k)$ is true for some positive integer $k$, i.e. $3 \mid k^{3}-k$. We need to show that $P(k+1)$ is true, i.e. $3 \mid (k+1)^{3}-(k+1)$.

    \textbf{归纳步骤:} 假设对于某个正整数$k$,$3 \mid k^{3}-k$。我们需要证明$P(k+1)$是正确的,即$3 \mid (k+1)^{3}-(k+1)$。

    By the inductive hypothesis, $3 \mid k^{3}-k$, that is, $k^{3}-k=3m$ for some integer $m$. Then we have: 

    由归纳假设,$3 \mid k^{3}-k$,也就是说$k^{3}-k=3m$,其中$m$是某个整数。那么我们有:

    \begin{align*}
        (k+1)^{3}-(k+1) &= k^{3}+3k^{2}+3k+1-k-1\\
        &= k^{3}-k+3k^{2}+3k\\
        &= 3m+3k^{2}+3k\\
        &= 3(m+k^{2}+k).
    \end{align*}
    Thus, $3 \mid (k+1)^{3}-(k+1)$.

    由此,$3 \mid (k+1)^{3}-(k+1)$。

    By the inductive axiom, $P(n)$ is true $\forall n \in \mathbb{N}$.

    由归纳公理,$P(n)$对于所有$n \in \mathbb{N}$都是正确的。
\end{sol}

According to the two examples above, we can make a small conclusion for the induction axiom:

根据上面的两个例子,我们现在可以对归纳公理做一个结论:

\begin{enumerate}
    \item \textbf{Step1.} State our predicate $P(n)$. 提出我们的假设陈述$P(n)$。
    \item \textbf{Step2.} Prove the base case $P(1)$ is true. 证明基础情况$P(1)$是正确的。
    \item \textbf{Step3.} Assume that $P(k)$ is true and prove that $P(k+1)$ is true, $\forall k \in \mathbb{N}$. 假设$P(k)$是正确的并证明$P(k+1)$是正确的,$\forall k \in \mathbb{N}$。
    \item \textbf{Step4.} Give conclusion: by the induction axiom, $P(n)$ is true $\forall n \in \mathbb{N}$. 得出结论:根据归纳公理,$P(n)$对于所有$n \in \mathbb{N}$都是正确的。
\end{enumerate}

Besides, as you can see, there are some shortcuts in the induction proof. 

除此之外,如你所见,用归纳证明的过程实际上也有一些缺点。

\begin{enumerate}
    \item Induction cann't help us find the answers of problems, it only helps us to prove the correctness of the answers. 归纳法并不能帮助我们找到问题的答案,它只能帮助我们验证答案的正确性。
    \item Induction is not good for us to understanding the problems, we cann't get more details or insight of the problems by induction. 归纳法并不适合我们理解问题,我们不能通过归纳法得到更多更多问题的细节或是问题内蕴的东西。
\end{enumerate}

\section{A Seemly Wired Example 一个看似奇怪的例子}

All right, we have seen some introduction sample questions which are easy to help us being familiar with induction. Now, let's see a "seemly" 
wired problem with a wrong induciton solution to help us understand depper about the axiom.

好的,我们已经见识过了一些简单的题目,这些例子有助于我们熟悉归纳法的基本思想。现在,让我们看一个“看似”有点奇怪的问题吧,
并且下面写的归纳证明实际上是错误的(读者可以看看能否自己找出错误之处),不过这样的例子实际上有助于我们更深入地理解这个公理。

\begin{prb}
    Prove that all horses are the same color.

    证明所有的马都是同一种颜色。
\end{prb}

\begin{sol}
    Let $P(n)$ be the statement/predicate that: all horses in a group of $n$ horses are the same color. We will prove $P(n)$ is true for all positive integers $n$.

    令$P(n)$表示这样一个陈述:一个$n$匹马的群体中的所有马都是同一种颜色。我们将证明对于所有正整数$n$,$P(n)$是正确的。

    \textbf{Base Case:} $P(1)$ is true, since there is only one horse in the group, it is the same color as itself.
    
    \textbf{基础情况:} $P(1)$是正确的,因为群体中只有一匹马,它的颜色和它自己是一样的。

    \textbf{Inductive Step:} Assume that $P(k)$ is true for some positive integer $k$, that is, all horses in a group of $k$ horses are the same color. We need to show that $P(k+1)$: all horses in a group of $k+1$ horses are the same color is true.

    \textbf{归纳步骤:} 假设对于某个正整数$k$,$P(k)$是正确的,即$k$匹马的群体中的所有马都是同一种颜色。我们需要证明$P(k+1)$是正确的:$k+1$匹马的群体中的所有马都是同一种颜色。

    By the inductive hypothesis, all horses in a group of $k$ horses are the same color. Let's mark it that: $H_{1}=H_{2}=\cdots=H_{k}$.
    
    由归纳假设,$k$匹马的群体中的所有马都是同一种颜色。我们标记为:$H_{1}=H_{2}=\cdots=H_{k}$。

    Now, we add $1$ more horse to the group, then we have a group of $k+1$. Since all horses in a group of $k$ horses are the same color,

    现在,我们再添加一匹马到群体中,那么我们就有了$k+1$匹马。由于$k$匹马的群体中的所有马都是同一种颜色,

    then we have $H_{1}=H_{2}=\cdots=H_{k}$ and $H_{2}=H_{3}=\cdots=H_{k+1}$, then $H_{1}=H_{k+1}$, which means all horses in a group of $k+1$ horses are the same color.

    那么我们有$H_{1}=H_{2}=\cdots=H_{k}$和$H_{2}=H_{3}=\cdots=H_{k+1}$,那么$H_{1}=H_{k+1}$,这意味着$k+1$匹马的群体中的所有马都是同一种颜色。

    By the inductive axiom, $P(n)$ is true $\forall n \in \mathbb{N}$.
    
    由归纳公理,$P(n)$对于所有$n \in \mathbb{N}$都是正确的。
\end{sol}

Unfortunately, the proof is wrong, not just because the problem itself is wrong even we use induction to get a true result, but the most important is that our usage of induction makes some mistakes.

不幸的是,这个证明是错误的,不仅仅是因为问题本身是错误的但我们却用归纳法得到了一个正确的结果,更重要的是我们在使用归纳法的过程中就犯了一些错误。

\textbf{Analysis:}

\textbf{分析:}

The mistake appears in the inductive step: pay your attention to the situation when $k=1$. In this case, we have only 2 horses in the group, and the inductive hypothesis is that all horses in a group of $1$ horse are the same color. 
This means we now only have $H_{1}=H_{1}$ and $H_{2}=H_{2}$, then if we make the next step as $H_{1}=H_{2}$, that may be not true. This the key point of the mistake what we made.

这个错误出现在归纳步骤中:请注意当$k=1$时的情况。在这种情况下,我们只有2匹马在群体中,而归纳假设是所有的1匹马的群体中的马都是同一种颜色。
这意味着我们现在只有$H_{1}=H_{1}$和$H_{2}=H_{2}$,那么如果我们下一步写成$H_{1}=H_{2}$,那可能是不正确的。这就是我们犯错误的关键点。

The wired point of this problem is that we know it is a wrong fact, but we get a true result by using correct induction(not correctly using induction:O).

这个问题的奇怪之处在于我们知道这是一个错误的事实(我们当然知道所有的马不可能是同一种颜色的),但我们却通过正确的归纳法得到了一个正确的结果(实际上是没有正确地使用归纳法:O)。

\textbf{And what can we learn from that?} 

It tells us that when we use induction method to proof a statement, we need to make sure that the inductive step is correct, and even we need to check the base case carefully.

这个“奇怪”的问题告诉我们,当我们使用归纳法来证明一个陈述时,我们需要确保归纳步骤是正确的,甚至我们需要仔细检查基础情况。

\section{Conclusion 结论}

After the above statements and analysis, we can make a conclusion for the induction axiom:

在上面的陈述和分析之后,让我们来更系统性地总结一下归纳公理:

\begin{enumerate}
    \item Induction Axiom 归纳公理
    \begin{enumerate}
        \item \textbf{Step1.} State our predicate $P(n)$. 提出我们的假设陈述$P(n)$。
        \item \textbf{Step2.} Prove the base case $P(1)$ is true. 证明基础情况$P(1)$是正确的。
        \item \textbf{Step3.} Assume that $P(k)$ is true and prove that $P(k+1)$ is true, $\forall k \in \mathbb{N}$. 假设$P(k)$是正确的并证明$P(k+1)$是正确的,$\forall k \in \mathbb{N}$。
        \item \textbf{Step4.} Give conclusion: by the induction axiom, $P(n)$ is true $\forall n \in \mathbb{N}$. 得出结论:根据归纳公理,$P(n)$对于所有$n \in \mathbb{N}$都是正确的。
    \end{enumerate}
    \item Some thoughts: 一些思考:
    \item [shortcuts不足]Induction always cann't help us find the answers of problems(sometimes maybe), it only helps us to show the correctness of the statement. 归纳法并不能帮助我们找到问题的答案(有时候可能可以),它只能帮助我们证明陈述的正确性。
    \item [attentions注]Induction axiom is true, but in proof process we need to make sure that the inductive step is correct(especially some particular steps). Moreover, we even need to check the base case carefully. 归纳公理是正确的,但在证明过程中我们需要确保归纳步骤是正确的(尤其是一些特殊的步骤,比如前几步)。此外,我们有时也需要仔细检查基础情况的正确性。
\end{enumerate}

Finally, I hope this article can help you to understand the induction axiom more deeply and use it more flexibly:).

最后,希望这篇文章能帮助你更深入地理解归纳公理并更好、更灵活地使用它:)。

\end{document}